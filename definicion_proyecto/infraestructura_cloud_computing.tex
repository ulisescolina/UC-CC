\documentclass{article}
\renewcommand\refname{Referencias}
\renewcommand{\contentsname}{Índice de Contenído}
\usepackage{graphicx}
\graphicspath{{img/}}
\usepackage{caption}
\usepackage{subcaption}
\usepackage{float}
\usepackage{csquotes}
\usepackage[spanish]{babel}
\usepackage[
  style=ieee,
]{biblatex}
\usepackage[
  colorlinks=false,
  bookmarks=true]{hyperref}
\addbibresource{infraestructura_cloud_computing.bib}

\title{\textsc{Infraestructura para la Nube}\\Proyecto de Extensión\\{\tt borrador }}
\author{\textsc{Ulises C. Ramirez} [\textit{ulisesrcolina@gmail.com}]}
\date{\today}
\begin{document}
\maketitle
\tableofcontents

\section{Introducción}%
\label{sec:introduccion}
En los últimos años, con el avance de internet, el uso de componentes basados
en la ``nube'' ganó
mucha popularidad, tanto así que en la actualidad tenemos servicios para
guardar archivos en la nube\cite{m-19, m-20, m-21},
servicios multimedia (audio\cite{m-1, m-2, m-3, m-4},
audio en vivo\cite{m-22},
video\cite{m-5, m-6, m-7},
video en vivo\cite{m-8},
imágenes\cite{m-9, m-10, m-11}) basados en la nube,
servicios para editar documentos de manera colaborativa en la nube
\cite{m-12, m-13, m-14, m-15},
servicios para jugar en la nube\cite{m-16, m-17, m-18}, por mencionar
algunos\footnote{Cabe destacar que la lista de servicios basados en la nube es
mayor a la expuesta aquí, y esta se encuentra en constante crecimiento}
de los que actualmente se encuentran disponibles, haciendo énfasis especialmente
en servicios que utilizan las personas ``comunes''.

Esta nueva ola de servicios
en la nube trae consigo múltiples desafíos en las áreas técnicas en cuanto al
hardware y software para el sustento de la infraestructura que es necesaria
para brindar el servicio, estos desafíos vienen en forma de aprovisionamiento
de poder de cómputo que posibilite la escalabilidad, mecanismos que permitan la
alta disponibilidad, herramientas que den soporte al procesamiento de la
creciente cantidad de datos, etc.

Así nacen y se popularizan diferentes herramientas que permiten el manejo del
poder de cómputo de diferentes máquinas interconectadas, tales como
\texttt{OpenStack}\footnote{\url{https://openstack.org}}, \texttt{Apache
Mesos}\footnote{ \url{http://mesos.apache.org}  },
\texttt{DC/OS}\footnote{ \url{https://dcos.io}  }, lista que sigue en constante
crecimiento al igual que la cantidad de servicios en la nube.



\section{Descripción del Proyecto}%
\label{sec:descripcion_del_proyecto}
En este proyecto se plantea la implementación de recursos de hardware y
software teniendo como objetivo crear un cluster de unidades de cómputo en el
módulo de Apóstoles.

\section{I+D}%
\label{sec:i_d}
La implementación de estos recursos darán lugar a que se creen proyectos en nuevas áreas de investigación
dentro del módulo, de esta manera implementación formal de los recursos de cómputo en el módulo darán lugar
a que proyectos de investigación o cátedras que requieran el uso de éstos,
los tengan disponibles.




\section{Formación de RRHH}%
\label{sec:formacion_de_rrhh}
% Apuntar a que el mercado lo esta implementando en gran medida, por lo que
% suma el hecho de que los alumnos/docentes se capaciten y puedan educar en el
% tema, y todo esto se da dentro de la universidad.
El uso de técnologías basadas en la nube, que permiten el escalado horizontal y
la alta disponibilidad --entre otras cuestiones--, ha ganado mucha popularidad
(y lo sigue haciendo\cite{harper2019}) a lo largo de los ultimos años\cite{m-23},
por lo que la formación de recursos humanos que estén capacitados en el uso de
éstas tecnologías emergentes tendran un gran valor agregado, alumnos y docentes
que deseen acceder a los recursos tendrán los mismos en las instalaciones de la
facultad, de otra manera, sería necesario el alquiler del poder de cómputo de
un tercero\cite{aws, gce, ibm-cloud}.

% todo no se si esto está bien, pero como idea poner en evidencia el hecho de que
% las universidades de la región no tienen eso, por lo que sería llamativo tener
% RRHH que esté capacitado en el tema


% === Bilbiografia === %
\newpage

\printbibliography[keyword={web-oficial},title={Referencias: Sitios Oficiales}]
\printbibliography[keyword={articulo}, title={Referencias: Artículos}]
\end{document}
